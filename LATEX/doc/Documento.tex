\documentclass[
	% -- opções da classe memoir --
	12pt,				% tamanho da fonte
	openright,			% capítulos começam em pág ímpar (insere página vazia caso preciso)
	%twoside,			% para impressão em verso e anverso. Oposto a oneside
	oneside,
	a4paper,			% tamanho do papel. 
	% -- opções da classe abntex2 --
	%chapter=TITLE,		% títulos de capítulos convertidos em letras maiúsculas
	%section=TITLE,		% títulos de seções convertidos em letras maiúsculas
	%subsection=TITLE,	% títulos de subseções convertidos em letras maiúsculas
	%subsubsection=TITLE,% títulos de subsubseções convertidos em letras maiúsculas
	% -- opções do pacote babel --
	english,			% idioma adicional para hifenização
	french,				% idioma adicional para hifenização
	spanish,			% idioma adicional para hifenização
	brazil				% o último idioma é o principal do documento
	]{abntex2}

% ---
% Pacotes básicos 
% ---
\usepackage{lmodern}			% Usa a fonte Latin Modern				
\usepackage[T1]{fontenc}		% Selecao de codigos de fonte.
\usepackage[utf8]{inputenc}		% Codificacao do documento (conversão automática dos acentos)
\usepackage{lastpage}			% Usado pela Ficha catalográfica
\usepackage{indentfirst}		% Indenta o primeiro parágrafo de cada seção.
\usepackage{color}				% Controle das cores
\usepackage{graphicx}			% Inclusão de gráficos
\usepackage{microtype} 			% para melhorias de justificação
\usepackage[font={small,it}]{caption}

% ---
		
% ---
% Pacotes adicionais, usados apenas no âmbito do Modelo Canônico do abnteX2
% ---
\usepackage{lipsum} 	% para geração de dummy text
% ---
\usepackage{float}
% ---
% Pacotes de citações
% ---
\usepackage[brazilian,hyperpageref]{backref}	 % Paginas com as citações na bibl
\usepackage[alf]{abntex2cite}	% Citações padrão ABNT




% define o caminho das imagens
\graphicspath{{img/}}

% --- 
% CONFIGURAÇÕES DE PACOTES
% --- 

% ---
% Configurações do pacote backref
% Usado sem a opção hyperpageref de backref
\renewcommand{\backrefpagesname}{Citado na(s) página(s):~}
% Texto padrão antes do número das páginas
\renewcommand{\backref}{}
% Define os textos da citação
\renewcommand*{\backrefalt}[4]{
	\ifcase #1 %
		Nenhuma citação no texto.%
	\or
		Citado na página #2.%
	\else
		Citado #1 vezes nas páginas #2.%
	\fi}%
% ---


\newcommand{\myPutImg}[4]{
    \begin{figure}[H]
            \includegraphics[width=\linewidth,keepaspectratio=true,scale=#2]{img/#1}
        \captionof{figure}{#3}
        \label{fig:#4}
    \end{figure}
}
\newcolumntype{C}[1]{>{\centering\let\newline\\\arraybackslash\hspace{0pt}}m{#1}}
\newcommand{\myTitulo}{Sistema X}
\newcommand{\myAltor}{Lucas Teles, Thiago Carreno}
\newcommand{\myAno}{2016}
\newcommand{\mySub}{Manual de utilização do sistema X}

% ---
% Informações de dados para CAPA e FOLHA DE ROSTO
% ---
\titulo{\myTitulo}
\autor{\myAltor}
\local{São Paulo -- Brasil}
\data{\myAno}

%\coorientador{Nome Completo}
\instituicao{%
  Wunderman
}
\tipotrabalho{Manual }
% O preambulo deve conter o tipo do trabalho, o objetivo, 
% o nome da instituição e a área de concentração 
\preambulo{\mySub}
% ---

% ---
% Configurações de aparência do PDF final

% alterando o aspecto da cor azul
\definecolor{blue}{RGB}{41,5,195}

% informações do PDF
\makeatletter
\hypersetup{
     	%pagebackref=true,
		pdftitle={\@title}, 
		pdfauthor={\@author},
    	pdfsubject={\imprimirpreambulo},
	    pdfcreator={LaTeX with abnTeX2},
		pdfkeywords={abnt}{latex}{abntex}{abntex2}{}, 
		colorlinks=true,       		% false: boxed links; true: colored links
    	linkcolor=blue,          	% color of internal links
    	citecolor=blue,        		% color of links to bibliography
    	filecolor=magenta,      		% color of file links
		urlcolor=blue,
		bookmarksdepth=4
}
\makeatother
% --- 

% --- 
% Espaçamentos entre linhas e parágrafos 
% --- 

% O tamanho do parágrafo é dado por:
\setlength{\parindent}{1.3cm}

% Controle do espaçamento entre um parágrafo e outro:
\setlength{\parskip}{0.2cm}  % tente também \onelineskip

% ---
% compila o indice
% ---
\makeindex
% ---

% ----
% Início do documento
% ----
\begin{document}


% Retira espaço extra obsoleto entre as frases.
\frenchspacing 

% ----------------------------------------------------------
% ELEMENTOS PRÉ-TEXTUAIS
% ----------------------------------------------------------
% \pretextual

% ---
% Capa
% ---
\imprimircapa
% ---

% ---
% Folha de rosto
% (o * indica que haverá a ficha bibliográfica)
% ---
% \imprimirfolhaderosto%*
% ---

% ---
% RESUMOS
% ---

% resumo em português
\setlength{\absparsep}{18pt} % ajusta o espaçamento dos parágrafos do resumo
%\begin{resumo}
   
 %\textbf{Palavras-chaves}: IDS,Rede,Internet
%\end{resumo}

% ---
% inserir lista de ilustrações (figuras)
% ---
%\pdfbookmark[0]{\listfigurename}{lof}
%\listoffigures*
%\cleardoublepage
% ---

% ---
% inserir lista de tabelas
% ---
%\pdfbookmark[0]{\listtablename}{lot}
%\listoftables*
%\cleardoublepage
% ---

% ---
% inserir lista de abreviaturas e siglas
% ---

\begin{siglas}
  \item[DNS] Domain Name Server
  \item[SMTP] Simple Mail Transfer Protocol
\end{siglas}

% ---

% ---
% inserir o sumario
% ---
\pdfbookmark[0]{\contentsname}{toc}
\tableofcontents*
\cleardoublepage
% ---

% ----------------------------------------------------------
% ELEMENTOS TEXTUAIS 
% ----------------------------------------------------------
\textual


\chapter{Cadastro de usuarios}

Esta tela é responsavel por cadastrar os usuarios, ela ira parecer como na figura \ref{fig:cad1}.

\myPutImg{cad.jpg}{1}{Print do cadastro}{cad1}


\section{Definiçaõ dos campos}

Os campos obrigatorios do cadastro de usuario são:

\begin{itemize}
    \item Campo 1
    \item Campo 2
    \item Campo 3
\end{itemize}

Descrição sobre os campos na tabela \ref{table:campos1}


\begin{table}[H]
    \centering
    \renewcommand{\arraystretch}{2}
    \caption{campos da tela de cadastro}
    \label{table:campos1}
    \begin{tabular}{|C{4cm}|C{3.5cm}|C{7cm}|}
        \hline
        \textbf{Nome} & \textbf{Tipo} & \textbf{Descrição} \\
        \hline
        Campo 1 & Texto & Campo para o nome \\
        \hline
        Campo 2 & Numero & Campo para o numero \\
        \hline
        Campo 3 & Data & Campo para a data de nascimento \\
        
        \hline
    \end{tabular}
\end{table}


\section{Regras da tela}

\lipsum[3] %gera texto random
Pode ser conferido em \cite{SharepointManual}

\chapter{Cadastro de Produtos}

Esta tela é responsavel por cadastrar os usuarios, ela ira parecer como na figura \ref{fig:cad2}.

\myPutImg{cad.jpg}{1}{Print do cadastro}{cad2}


\section{Definiçaõ dos campos}

Os campos obrigatorios do cadastro de usuario são:

\begin{itemize}
    \item Campo 1
    \item Campo 2
    \item Campo 3
\end{itemize}

Descrição sobre os campos na tabela \ref{table:campos2}


\begin{table}[H]
    \centering
    \renewcommand{\arraystretch}{2}
    \caption{campos da tela de cadastro}
    \label{table:campos2}
    \begin{tabular}{|C{4cm}|C{3.5cm}|C{7cm}|}
        \hline
        \textbf{Nome} & \textbf{Tipo} & \textbf{Descrição} \\
        \hline
        Campo 1 & Texto & Campo para o nome \\
        \hline
        Campo 2 & Numero & Campo para o numero \\
        \hline
        Campo 3 & Data & Campo para a data de nascimento \\
        
        \hline
    \end{tabular}
\end{table}


\section{Regras da tela}

\lipsum[3-5] % gera texto random
Pode ser conferido em \cite{ExcelManual}




\vspace{1.25cm}


% ----------------------------------------------------------
% ELEMENTOS PÓS-TEXTUAIS
% ----------------------------------------------------------
\postextual
% ----------------------------------------------------------

% ----------------------------------------------------------
% Referências bibliográficas
% ----------------------------------------------------------
\bibliography{Reference}


% Revisao Bibliografica
\end{document}
